\documentclass[12pt,a4paper,titlepage,ngerman,pdftex]{report}
\usepackage[utf8]{inputenc}
\usepackage[german]{babel}
\usepackage[T1]{fontenc}
\usepackage{amsmath}
\usepackage{amsfonts}
\usepackage{amssymb}
\usepackage{graphicx}
\usepackage{acronym}
\usepackage{setspace}
\usepackage{geometry}
\usepackage{caption}
\usepackage{subcaption}
\usepackage{hyperref}
\usepackage{listings}
\geometry{a4paper, top=25mm, left=25mm, right=25mm, bottom=25mm,
headsep=10mm, footskip=12mm}



\begin{document}
%%%%%%%%%%%%%%%%%%%%%%%%%%%%%%%%%%%%%%%%%%%%%%%
%%%%%%%%%%%%%%%%%    TITLE    %%%%%%%%%%%%%%%%%
%%%%%%%%%%%%%%%%%%%%%%%%%%%%%%%%%%%%%%%%%%%%%%%
\begin{titlepage}
	\centering
	%\includegraphics[width=0.15\textwidth]{pic/picsim_logo_full.png}\par\vspace{1cm}
	{\scshape\LARGE Duale Hochschule Baden-Württemberg \par}
	\vspace{1cm}
	{\scshape\Large Dokumentation\par}
	\vspace{1.5cm}
	{\huge\bfseries Bierwart\par}
	\vspace{2cm}
	{\Large\itshape Levin Baumann, Marius Holwein\par}
	\vfill
	Dozent\par
	Daniel \textsc{Lindner}

	\vfill

% Bottom of the page
	{\large \today\par}
\end{titlepage}

%%%%%%%%%%%%%%%%%%%%%%%%%%%%%%%%%%%%%%%%%%%%%%%
%%%%%%%%%%%%%%%%%  LISTINGS  %%%%%%%%%%%%%%%%%
%%%%%%%%%%%%%%%%%%%%%%%%%%%%%%%%%%%%%%%%%%%%%%%
\pagenumbering{Roman} 
\tableofcontents
\listoffigures

%%%%%%%%%%%%%%%%%%%%%%%%%%%%%%%%%%%%%%%%%%%%%%%
%%%%%%%%%%%%%%%%%  Acronym  %%%%%%%%%%%%%%%%%
%%%%%%%%%%%%%%%%%%%%%%%%%%%%%%%%%%%%%%%%%%%%%%%
\chapter*{Abkürzungsverzeichnis}
\begin{acronym}[all]
	\acro{dhbw}[DHBW]{Duale Hochschule Baden-Württemberg}
	\acro{rest}[REST]{REstful State Transfer}
	\acro{api}[API]{Application Programming Interface}
\end{acronym}
\onehalfspacing


\chapter{Einleitung}\label{einleitung}
\pagenumbering{arabic}
Lorem Ipsum, blah blah blah

\chapter{Projektbeschreibung}
Der Bierwart, im Rahmen dieser Abgabe, ist eine reine \ac{rest}-\ac{api}. Es besteht ein separat ausführbares Frontend unter \url{https://github.com/HerrLevin/Bierwart-frontend}.
Die \ac{api} ist mit Swagger dokumentiert und befindet sich unter \url{http://localhost:8000/swagger/}\footnote{nur nach erfolgreicher Einrichtung beschrieben in \ref{sec:einrichtung} möglich.}.
Hier lassen sich mit Betätigung des Buttons \textit{Try it out} einzelne Requests ohne Zusatzprogramme ausführen.

\section{Einrichtung}\label{sec:einrichtung}

\begin{enumerate}
    \item \texttt{git clone https://github.com/HerrLevin/Bierwart.git}
    \item \texttt{cd Bierwart}
    \item \texttt{composer install}
    \item sqlite-Datenbank in \texttt{database} namens \texttt{database.sqlite} erstellen und\\ \texttt{default\_migration.sql} ausführen
        \begin{itemize}
            \item alternativ \texttt{database.bak.sqlite} in \texttt{database.sqlite} umbenennen
        \end{itemize}
    \item \texttt{php -S localhost:8000} im Hauptverzeichnis
    \item \url{http://localhost:8000/swagger/} aufrufen
\end{enumerate}


\end{document}