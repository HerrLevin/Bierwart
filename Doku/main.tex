\documentclass[12pt,a4paper,titlepage,ngerman,pdftex]{report}
\usepackage[utf8]{inputenc}
\usepackage[german]{babel}
\usepackage[T1]{fontenc}
\usepackage{amsmath}
\usepackage{amsfonts}
\usepackage{amssymb}
\usepackage{graphicx}
\usepackage{acronym}
\usepackage{setspace}
\usepackage{geometry}
\usepackage{caption}
\usepackage{subcaption}
\usepackage{hyperref}
\usepackage{listings}
\usepackage{textcomp}
\geometry{a4paper, top=25mm, left=25mm, right=25mm, bottom=25mm,
    headsep=10mm, footskip=12mm}
\usepackage{xcolor}

\definecolor{mygreen}{rgb}{0,0.6,0}
\definecolor{mygray}{rgb}{0.5,0.5,0.5}
\definecolor{mymauve}{rgb}{0.58,0,0.82}
\definecolor{backcolour}{rgb}{0.827, 0.827, 0.827}
\definecolor{dkblue}{rgb}{0,0,.6}
\definecolor{dkyellow}{cmyk}{0,0,.8,.3}

\lstset{
    language        = php,
    basicstyle      = \small\ttfamily,
    keywordstyle    = \color{dkblue},
    stringstyle     = \color{red},
    identifierstyle = \color{mygreen},
    commentstyle    = \color{gray},
    emph            =[1]{php},
    emphstyle       =[1]\color{black},
    emph            =[2]{if,and,or,else},
    emphstyle       =[2]\color{dkyellow}}

\lstset{ %
    backgroundcolor=\color{backcolour},   % choose the background color
    basicstyle=\footnotesize,        % size of fonts used for the code
    breaklines=true,                 % automatic line breaking only at whitespace
    captionpos=b,                    % sets the caption-position to bottom
    commentstyle=\color{mygreen},    % comment style
    escapeinside={\%*}{*)},          % if you want to add LaTeX within your code
    keywordstyle=\color{blue},       % keyword style
    stringstyle=\color{mymauve},     % string literal stylebreakatwhitespace=false,
    keepspaces=true,
    numbers=left,
    numbersep=5pt,
    showspaces=false,
    showstringspaces=true,
    showtabs=false,
    tabsize=2
}

\begin{document}
%%%%%%%%%%%%%%%%%%%%%%%%%%%%%%%%%%%%%%%%%%%%%%%
%%%%%%%%%%%%%%%%%    TITLE    %%%%%%%%%%%%%%%%%
%%%%%%%%%%%%%%%%%%%%%%%%%%%%%%%%%%%%%%%%%%%%%%%
    \begin{titlepage}
        \centering
        %\includegraphics[width=0.15\textwidth]{pic/picsim_logo_full.png}\par\vspace{1cm}
        {\scshape\LARGE Duale Hochschule Baden-Württemberg \par}
        \vspace{1cm}
        {\scshape\Large Dokumentation\par}
        \vspace{1.5cm}
        {\huge\bfseries Bierwart\par}
        \vspace{2cm}
        {\Large\itshape Levin Baumann, Marius Holwein\par}
        \vfill
        Dozent\par
        Daniel \textsc{Lindner}

        \vfill

% Bottom of the page
        {\large \today\par}
    \end{titlepage}

%%%%%%%%%%%%%%%%%%%%%%%%%%%%%%%%%%%%%%%%%%%%%%%
%%%%%%%%%%%%%%%%%  LISTINGS  %%%%%%%%%%%%%%%%%
%%%%%%%%%%%%%%%%%%%%%%%%%%%%%%%%%%%%%%%%%%%%%%%
    \pagenumbering{Roman}
    \tableofcontents
    \listoffigures

%%%%%%%%%%%%%%%%%%%%%%%%%%%%%%%%%%%%%%%%%%%%%%%
%%%%%%%%%%%%%%%%%  Acronym  %%%%%%%%%%%%%%%%%
%%%%%%%%%%%%%%%%%%%%%%%%%%%%%%%%%%%%%%%%%%%%%%%
    \chapter*{Abkürzungsverzeichnis}
    \begin{acronym}[all]
        \acro{dhbw}[DHBW]{Duale Hochschule Baden-Württemberg}
        \acro{rest}[REST]{REstful State Transfer}
        \acro{api}[API]{Application Programming Interface}
    \end{acronym}
    \onehalfspacing


    \chapter{Einleitung}\label{ch:einleitung}
    \pagenumbering{arabic}
    Lorem Ipsum, blah blah blah


    \chapter{Projektbeschreibung}\label{ch:projektbeschreibung}
    Der Bierwart, im Rahmen dieser Abgabe, ist eine reine \ac{rest}-\ac{api}. Es besteht ein separat ausführbares Frontend unter \url{https://github.com/HerrLevin/Bierwart-frontend}.
    Die \ac{api} ist mit Swagger dokumentiert und befindet sich unter \url{http://localhost:8000/swagger/}\footnote{nur nach erfolgreicher Einrichtung beschrieben in \ref{sec:einrichtung} möglich.}.
    Hier lassen sich mit Betätigung des Buttons \textit{Try it out} einzelne Requests ohne Zusatzprogramme ausführen.


    \section{Einrichtung}\label{sec:einrichtung}

    \begin{enumerate}
        \item \texttt{git clone https://github.com/HerrLevin/Bierwart.git}
        \item \texttt{cd Bierwart}
        \item \texttt{composer install}
        \item sqlite-Datenbank in \texttt{database} namens \texttt{database.sqlite} erstellen und\\ \texttt{default\_migration.sql} ausführen
        \begin{itemize}
            \item alternativ \texttt{database.bak.sqlite} in \texttt{database.sqlite} umbenennen
        \end{itemize}
        \item \texttt{php -S localhost:8000} im Hauptverzeichnis
        \item \url{http://localhost:8000/swagger/} aufrufen
    \end{enumerate}


    \chapter{Entwicklung}\label{ch:entwicklung}
    %Idk, was hier alles rein soll


    \section{Refactoring}\label{sec:refactoring}

    \subsection{Router}\label{subsec:router}
    Der Router des Projektes funktioniert nach einem "First Pattern Matching"-Prinzip.
    Um dies in der Routenkonfiguration leicht umzusetzen, beendet sich die Ausführung des Prozesses, sobald eine Route gefunden und die entsprechende Methode darin ausgeführt wurde.
    Dies geschah anfangs über ein \verb|exit(0);|, was leider untestbar ist.
    Der hier entsprechende Codesmell sind \href{https://refactoring.guru/refactoring/technical-debt}{technische Schulden}, da dieses Codefragment zunächst nur als ungetestetes MVP diente und die Folgen der \glqq Untestbarkeit\grqq{} zunächst nicht bewusst waren.
    Um dieses Problem zu umgehen, ersetzten wir zunächst diese Zeile durch das Werfen einer eigenen Exception: \verb|throw new TaskFailedSuccessfullyException();|\footnote{\url{https://github.com/HerrLevin/Bierwart/commit/b56b8ea86c12fe04e9f5ead235ca783f14aa1910}}

    Diese konnten wir sowohl in unseren Tests abfangen, als auch in unserer \verb|index.php|:

    \begin{lstlisting}[language=php,label={lst:lstlisting}]
$request = $_SERVER['REQUEST_URI'];
$router = new Router($request);
try {
    $router->get('/', Bierwart::class, 'getHelloWorld');
    $router->get('/generateReport', ReportsController::class, 'generateReport');

    Router::abort();
} catch (TaskFailedSuccessfullyException) {
//This place is a message... and part of a system of messages... pay attention to it!
//Sending this message was important to us. We considered ourselves to be a powerful culture.
//This place is not a place of honor... no highly esteemed deed is commemorated here... nothing valued is here.
//What is here was dangerous and repulsive to us. This message is a warning about danger.
//The danger is in a particular location... it increases towards a center... the center of danger is here... of a particular size and shape, and below us.
//The danger is still present, in your time, as it was in ours.
//The danger is to the body, and it can kill.
//The form of the danger is an emanation of energy.
//The danger is unleashed only if you substantially disturb this place physically. This place is best shunned and left uninhabited.
}
    \end{lstlisting}
\noindent
    Im weiteren Verlauf, wurde die statische Methode \verb|Helpers::dd();| eingeführt.
    Diese steht für \textit{Dump \& Die} -- sie gibt, sofern mitgegeben, einen Dump einer Variable aus und beendet den aktuellen Prozess.
    Dies hat den Vorteil zur Folge, dass der Aufruf der statischen Methode in Tests mockbar ist.

    Dummerweise hat somit die \verb|TaskFailedSuccessfullyException| dieselbe funktionalität wie \verb|dd()|.
    Der hierfür entsprechende Codesmell ist die \href{https://pragmaticways.com/31-code-smells-you-must-know/#6_Oddball_Solution}{Oddball Solution}.
    Deshalb haben wir die ungewöhnlichere und technisch teurere Funktionalität der Exception ausgetauscht durch \verb|dd()|.\footnote{\url{https://github.com/HerrLevin/Bierwart/commit/20e8c915281df7b791abdbded9d593fbce22344f}}

    In beiden Fällen wurde hierfür das \href{https://refactoring.guru/substitute-algorithm}{Substitute Algorithm}-Prinzip zum Refactoring genutzt.

    \subsection{DB::class}\label{subsec:db::class}
    Weil wir am Anfang nicht richtig wussten, wo wir hin gehen mit unserem Code, bauten wir eine Klasse zum Kommunizieren mit der Datenbank: \verb|DB::class|
    Da wir in einer super Vorlesung gelernt haben, dass es deutlich bessere Wege gibt, haben wir einen Query-Builder gebaut, der diese schlechte Klasse ablöst.
    Das Ganze ist ein \href{https://refactoring.guru/smells/dead-code}{Dead Code} CodeSmell, deshalb haben wir die Klasse einfach gelöscht.\footnote{\url{https://github.com/HerrLevin/Bierwart/commit/c8465100f3ea649a7e67be5ef4d1282134180e82}}


\end{document}